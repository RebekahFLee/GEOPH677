%% cv-us.tex
%
% Curriculum Vitae
%
% You may use use this document as a template to create your own CV
% and you may redistribute the source code freely. No attribution is
% required in any resulting documents, however, I do ask that you
% please leave this notice and the above URL in the source code if you
% choose to redistribute this file.
%
% $Id: cv-us.tex,v 1.23 2006/06/21 00:14:55 jrblevin Exp $
%
%%---------------------------------------------------------------------------
%
% Notes:
%
% * To create a new page use
%   \newpage \opening
%
% * res.cls includes an \address{} command which can be used up to twice,
%   but my address is too long for the format it uses.
%
% * Alternate documentclass statement to put headings in margin:
%   \documentclass[margin,line,11pt,final]{res}
%
% * Can divide publication/presentation list into subsections by year:
%   \section{\bf\small\hspace{8mm}2006}
%
%%----------------------------------------------------------------------------

%\documentclass[overlapped,line,final]{res}
\documentclass[overlapped,line,letterpaper]{res}
%\documentclass[margin,line,11pt,final]{res}

\usepackage{ifpdf}
\usepackage{helvet}
%\usepackage{times}
%\usepackage{setspace}

\ifpdf
  \usepackage[pdftex]{hyperref}
\else
  \usepackage[hypertex]{hyperref}
\fi

\hypersetup{
  letterpaper,
  colorlinks,
  urlcolor=blue,
  pdfpagemode=none, 
  pdftitle={title},
  pdfauthor={author},
  pdfcreator={},
  pdfsubject={subject},
%  pdfkeywords={economics microeconomics econometrics empirical urban
%    game theory applied mathematics differential equations TANH method
%    hyperbolic tangent centroid decomposition updating singular value
%    decomposition linear algebra microeconomics linux unix german}
}

\def\Cplusplus{{\rm C\raise.5ex\hbox{\small ++}}}

%%===========================================================================%%

\pagestyle{myheadings}


\begin{document}


%---------------------------------------------------------------------------
% Document Specific Customizations

% Make lists without bullets and with no indentation
\setlength{\leftmargini}{0em}
\renewcommand{\labelitemi}{}

% Use large bold font for printed name at top of pages
\renewcommand{\namefont}{\large\textbf}


\newcommand{\bME}{{\bf T. D. Mikesell}}



%---------------------------------------------------------------------------

\name{HW1: Instrument Responses and Deconvolution \hspace{3cm} Rebekah Lee}
\begin{resume}
\vspace{-.4cm}\hspace{-1.3cm}{\bf GEOPH677}
\hspace{11.2cm}{\bf Jan 30 2017}

\subsection{\hspace{-1.3cm}Part 1: An electromagnetic velocity sensor}

\section{A) Explain the Biot-Savart law.}

The Biot-Savart law gives the magnetic field of a steady line current and is analogous to Coulomb's law in electrostatics:

$${\bf B}({\bf r}) = \frac{\mu_0}{4\pi} \int \frac{{\bf I \times \hat{r'} }}{r'^2}dl = \frac{\mu_0}{4\pi}I \int \frac{d{\bf l \times \hat{r'} }}{r'^2}dl$$

Where ${\bf I}$ is the electric current, ${\bf r'}$ is the vector from the source to the point ${\bf r}$, dl is an element along the wire in the direction of the current and $\mu_0 = 4\pi \times 10^{-7} N/A^2$ is the permeability of free space. The units for the magnetic field are newtons per ampere-meter (N/ (A m)) or teslas (T) and the integral is along the current path in the same direction as the flow.
 

Source: Griffiths, David. Introduction to Electrodynamics. Fourth Edition

\section{B) Explain Onsager's reciprocal theorem}

Onsager's reciprocal theorem involves the reciprocity of coupled electrical and thermal systems. Lars Onsager's 1931 paper uses the following notation:


$X_1$ and $X_2$ are driving electrical and thermodynamic forces, respectively and, if they were independent, could be written as:

$$X_1 = R_1J_1$$
$$X_2 = R_2J_2$$


where $R_1$ and $R_2$ are the electrical resistivity and thermal resistance, respectively. $J_1$ and $J_2$ are electrical and thermal current.

However, these systems are coupled since electrical current is not independent of the temperature. Therefore we can add in this dependency using cross coefficients $R_{12}$ and $R_{21}$ so that:

$$X_1 = R_{11}J_1 + R_{12}J_2,$$ and

$$X_2 = R_{21}J_1 + R_{22}J_2.$$ 

My understanding of this is that the electrical force is equal to electrical resistance times current density plus the current  scaled by a coeffecient that represents the coupling of the thermal heat on the electrical system.

Similarly, the thermodynamic force is equal to the thermal resistance times thermal current density plus the electrical current density scaled by a coeffecient representing the coupling of the electrical system on the thermal. The reciprocity theorem states that these two coefficients are equal,

$R_{12}=R_{21}$

Sources:

Onsager,Lars. Reciprocal Relations in Irreversible Processes. I. Physical Review, Vol 37. 1931

\begin{verbatim}http://www.iue.tuwien.ac.at/phd/holzer/node24.html\end{verbatim} 
\section{C)How would you compute $l$ given a coil with radius $r$ and number of coils $n$?}


$l = 2\pi rn $

\section{D) Derive equation 12.49}

First list the forces by looking at Figure 12.15. 
\begin{enumerate}
	\item Force of gravity on the mass
	\item Force from the spring
	\item Magnetic force 
\end{enumerate}

So using Newton, we have:
$F = F_{gravity} -F_{spring} - F_{mag}$

We saw in class that the force from the spring $(K(z-l_0))$ and the force of gravity ($Mg =k(z-l_0) $) combine and can be written as kz(t). We also saw that there are two sources of acceleration, from the seismometer and from the ground, so that we have:

\begin{equation}
$$$$z'' +u'' = -\omega_0^2z -\frac{1}{M}F{mag}. $$$$
\end{equation}

Rearranging and dividing by M:

\begin{equation}
$$$$z'' +u'' = -\omega_0^2z -\frac{1}{M}F{mag}, $$$$
\end{equation}

where $w_o^2=\frac{k}{m}$.


Aki and Richards give the magnetic force (from the Lorentz force law) as:

$F=IlB$,

where I is the current, l is the length of the wire and B is the flux density. The mechanical power is the rate that work is done so they multiply both sides by the velocity of the moving mass, or z'. so that : 

$Fz' = IlBz'$

Then they say that the mechanical power must be consumed by the resistance and since electrical power is equal to $P = VI$ (by Ohm's law) they obtain

 $V = lBz'$.

 Now they define lB as G so $V=Gz'$. Using Ohm's law and the total resistance equal to $R_0 + R$, $I(R_0 +R) = Gz'$ and simple division yields

\begin{equation}
	 I = \frac{Gz'}{R_0 +R}.
	 \label{eqn1}
\end{equation}

Now $F = GI$ and we can substitute in equation (3) to get

$F = \frac{G^2z'}{R_0+R}$,

and in turn substitute this back into equation (2) so that:

\begin{equation}
	z'' + u'' = -\omega_0^2 z - \frac{G^2}{R_0+R}\frac{z'}{M}
\end{equation}
%---------------------------------------------------------------------
	
or

\begin{equation}
	z'' + \omega_0^2 z  = -u'' - \frac{G^2}{R_0+R}\frac{z'}{M}.
\end{equation}

In class we saw that for a similar system with a dashpot instead of a magnet we get: 


\begin{equation}
	z'' + \omega_0^2 z  = -u'' - 2\epsilon z'.
\end{equation}

and comparing the two equations:

\begin{equation}
	2\epsilon = \frac{G^2}{R_0+R}\frac{1}{M}
\end{equation}

This equation relates the two dampening terms in the equation of motion for both systems. If there is also mechanical attenuation, $\epsilon_{0}$ then

\begin{equation}
	\epsilon = \epsilon_0 + \frac{G^2}{2(R_0+R)}\frac{1}{M}.
\end{equation}
%---------------------------------------------------------------------

\section{E) When was the electromagnetic sensor introduced in seismology and who introduced it?}

The electromagnetic sensor was introduced in 1914 by Russian scientist Galitzin.
%--------------------------------------------------------------------------------------------


\end{resume}
\end{document}
%----------------------------------------------------------------------
